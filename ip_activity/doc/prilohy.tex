%=========================================================================
% (c) Kateřina Pilátová, 3BIT
\setlength{\parindent}{0em}
\setlength{\parskip}{1em}

\csdoublequotesoff
\chapter{CD Content}
This section is concerned with the contents of the attached CD.
It includes the basic set of files forming the developed system.

However, not all files needed for running the system are included -- the
autotool configuration file and IP address library for C++ which was
extended for the purposes of this thesis are available in the git
repository forks (see Section~\ref{sec:requirements}).

The main sections are:
\begin{itemize}
  \item \itab{\footnotesize{\texttt{.}}} \tab{The root directory contains source files, scripts and guides.}
  \item \itab{\footnotesize{\texttt{images/}}} \tab{In this directory, visualised images are to be stored.}
  \item \itab{\footnotesize{\texttt{tests/}}} \tab{This directory is dedicated to testing and contains test suite.}
  \item \itab{\footnotesize{\texttt{doc/}}} \tab{Here the the source files for the thesis text version are located.}
\end{itemize}

Even though most of the files contain a~section with the used~licence, \textit{licence.txt} file is also included.


\chapter{Installation Guide}
\section{Requirements}\label{sec:requirements}
In order to use the system, the following requirements must be met:
\begin{itemize}
   \item The target system must have g++ installed.
   \item The target system must have Python 2 or 3 available.
   \item In order to run the web client, the target browser must have JavaScript enabled.
   \item In order to install the backend, access to Nemea repository is required.
   \item It is essential to download the updates of the Nemea repository and Nemea framework
   from my forked repository of Nemea-Framework~\cite{my_repo_framework} and Nemea-Modules~\cite{my_repo_modules}.
   In there, changes of autotools configuration file and \textit{ipaddr\_cpp.h} are up-to-date.
\end{itemize}

\section{Additional Packages}
\begin{itemize}
   \item For backend, only C++ \textit{Boost} library and \textit{yaml-cpp} version 0.5.3 libraries are required.
   In order to run the backend on a~machine (for example \textit{benefizio}) where the 0.5.3 version is not available,
   the library has to be installed straight from the source~\cite{yaml_cpp} and the path to it included in the
   \textit{Makefile.am}.
   \item The attached archive contains a~script \textit{install\_dependencies.sh} which should
   install the required packages for both C++ (if the 0.5.3 version is available) and Python both 2.x and 3.x.
\end{itemize}

\section{Running the System}
The backend has to be run first so that it can create the bitmaps and more importantly, the configuration file.
The backend parameters are discussed in Section~\ref{sec:backend_params} and can be displayed by
entering \texttt{-h [trap]} parameter.

The server program cannot be executed until the configuration file is created. Server's parameters
are listed in Section~\ref{sec:server_params} or can be displayed by entering \texttt{-h} or \texttt{-{}-help}
parameter. When executing the server, make sure to set up the port based on your options.

The frontend program has to run in a browser which has \textit{JavaScript} enabled. It does not need
an internet connection as the \textit{jQuery} library is provided in the archive.
